\chapter{外文资料原文}
\label{cha:engorg}

\title{The title of the English paper}

\textbf{Abstract:} As one of the most widely used techniques in operations
research, \emph{ mathematical programming} is defined as a means of maximizing a
quantity known as \emph{bjective function}, subject to a set of constraints
represented by equations and inequalities. Some known subtopics of mathematical
programming are linear programming, nonlinear programming, multiobjective
programming, goal programming, dynamic programming, and multilevel
programming$^{[1]}$.

It is impossible to cover in a single chapter every concept of mathematical
programming. This chapter introduces only the basic concepts and techniques of
mathematical programming such that readers gain an understanding of them
throughout the book$^{[2,3]}$.


\section{Single-Objective Programming}
Contents

\section*{References}
\begin{originbbl}
  \item M. Addlesee et al., ``Implementing a sentient computing system,'' IEEE Comput., vol. 34, no. 8, pp. 50–56, Aug. 2001.
  \item K. Al Nuaimi and H. Kamel, ``A survey of indoor positioning systems and algorithms,'' in Proc. IEEE IIT, 2011, pp. 185–190.
  \item P. Bahl, V. N. Padmanabhan, and A. Balachandran, ``Enhancements to the radar user location and tracking system,'' Microsoft Corp., Redmond, WA, USA, Tech. Rep., 2000. 
\end{originbbl}
